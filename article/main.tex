\documentclass[12pt]{ctexart} % 使用 ctexart 文档类,支持中文排版

% 宏包引入
% \usepackage{ctex} % <<<【已删除】ctexart 文档类已包含 ctex 宏包,重复加载会导致冲突。
\usepackage{setspace} % 设置行间距
\usepackage{indentfirst} % 首行缩进

% --- 交叉引用和参考文献处理 ---
% 【关键】hyperref 必须在 cite 之前加载
\usepackage{hyperref} 
\usepackage{bookmark} % 解决 'Rerun to get outlines right' 警告

\usepackage[numbers]{natbib} % 使用 natbib 宏包,并使用数字样式
% 注意:这里从 cite 换成了 natbib,它与手动列表(thebibliography)配合更好,并且支持 \citep, \citet 等。
% 如果您坚持使用 \upcite,则需要手动将 natbib 样式调整为上标格式。
% 考虑到您希望简化,我们使用 natbib 的标准数字引用。
\renewcommand{\cite}[1]{\textsuperscript{\citep{#1}}} % 重新定义 \cite 为上标数字样式
\newcommand{\upcite}[1]{\textsuperscript{\citep{#1}}} % 保持原有 \upcite 定义
\usepackage{graphicx} % 插入图片
\usepackage{amsmath} % 数学公式
\usepackage{amsfonts} % 数学字体
% ...
\usepackage{amssymb} % 数学符号
\usepackage{fancyhdr} % 自定义页眉和页脚
\usepackage{zhlipsum}


% 导言区设置
\usepackage[includeheadfoot]{geometry} % 包含页眉页脚在计算范围内
\geometry{
    a4paper,
    left=2.5cm,
    right=2.5cm,
    top=1.5cm,
    bottom=2cm,
    headsep=0.5cm,
    footskip=1cm
}

\usepackage{titlesec} % 控制标题样式

\ctexset{
    section = {
        name = {第,章}, % 设置编号显示为“第x章”
        number = \chinese{section} % 使用中文数字编号
  }
}

% 字体设置
\ctexset{
    section/format={\centering\heiti\zihao{-2}}, % 一级标题:黑体三号,居中
    subsection/format={\heiti\zihao{-3}} % 二级标题:黑体四号
}
% \setmainfont{SimSun} % (pdflatex 不支持,改用 ctex 默认字体映射)
\zihao{-4} % 设置正文字号为小四

% 行间距设置
\onehalfspacing % 1.5 倍行间距

% 设置页眉页脚样式
\fancyhf{} % 清除所有页眉和页脚
\fancyfoot[C]{\thepage} % 在页脚居中显示页码

\usepackage{tocloft}
\renewcommand{\cftsecleader}{\cftdotfill{\cftdotsep}}
\renewcommand{\cftdotsep}{1}
\renewcommand{\cfttoctitlefont}{\hfill\heiti\zihao{-2}}
\renewcommand{\cftaftertoctitle}{\hfill}

% \setCJKmainfont[BoldFont=SimHei]{SimSun} % (pdflatex 不支持)


% 文档开始
\begin{document}

% 标题页:强制无页码
\pagenumbering{gobble} % 禁用所有页码
\begin{figure}
    \centering
    \includegraphics[width=0.7\textwidth]{fig/logo.png} % 这里换成你的图片文件名
    \label{fig:myimage}
\end{figure}

\title{%
    \heiti\zihao{-2} 习近平总书记给东北大学全体师生的重要回信学习体会%
    \vspace{6\baselineskip} % 插入 3 行空白
}
\author{
\begin{tabular}{l}
    姓名:李狗蛋 \\
    学号:114514 \\
    专业:土木工程 \\
    学院:学院 \\
\end{tabular}
}
\date{2025年9月}
\maketitle

% 设置摘要和目录页码为大写罗马数字
\clearpage % 确保摘要单独起页
\pagenumbering{Roman} % 重新开启并设置为大写罗马数字
\pagestyle{fancy} % 重新启用页脚样式
\section*{摘要}
    \addcontentsline{toc}{section}{摘要} % 可选:让摘要显示在目录里
    \songti\zihao{-4} % 确保摘要部分也是宋体小四
  \ 东B
  
  东北大学建校百年之际,习近平总书记的重要回信充分肯定了学校“为党育人、为国育才”的办学初心和使命,强调了大学教育在国家发展中的基础性、战略性地位。文章围绕回信精神,梳理东北大学在救国、建国、强国等不同时期的贡献,结合新时代教育强国战略,探讨大学如何在立德树人、人才培养、服务国家发展等方面发挥更大作用。同时,阐述新时代大学生的责任与担当,强调青年应将个人理想与国家发展紧密结合,以实现民族复兴伟业。研究旨在揭示大学教育与国家发展的内在联系,为新时代教育改革与人才培养提供参考。

    \textbf{关键词}:东北大学;习近平回信;大学教育;国家发展;人才培养

\clearpage
% 目录
\tableofcontents

% 设置正文页码为阿拉伯数字,并从1开始重新计数
\clearpage % 确保正文从新页开始
\pagenumbering{arabic} % 切换为阿拉伯数字并重置计数器
\pagestyle{fancy} % 重新启用页脚样式

% 正文
\section{引言}
    \label{sec:intro}
    习近平总书记在东北大学建校百年之际的重要回信,不仅是对学校百年历程的肯定,更是对新时代大学教育的深刻指引。回信中强调的“为党育人、为国育才”的办学初心,深刻揭示了高等教育与国家命运的紧密联系。本文旨在深入学习和领会回信精神,探讨新时代背景下,高等教育如何更好地服务于国家战略,培养担当民族复兴大任的时代新人。
\clearpage % 清除浮动,另起一页
\section{东北大学百年发展与使命传承}
    \label{sec:second}
    东北大学自1923年建校以来,始终高扬爱国主义旗帜,与国家命运紧密相连。作为中国最早一批由国家创办的综合性大学之一,东北大学不仅承担着传播科学知识、培养人才的使命,更在民族危亡、社会动荡之际,发挥了救亡图存的重要作用。无论是在抗日战争的烽火中流亡办学,还是在新中国成立后积极投身社会主义建设,东北大学始终以“育人兴邦”为核心理念,为国家发展和民族复兴贡献力量。

    习近平总书记在致东北大学师生的回信中明确指出,东北大学“始终以育人兴邦为使命,形成了鲜明办学特色,培养了大批优秀人才,为国家、为民族作出了积极贡献”\upcite{ref1}。这一评价不仅是对学校百年奋斗历程的高度肯定,也是对新时代东北大学继续前行的殷切期望。东北大学的百年历史正是中国高等教育与国家发展紧密结合的缩影,折射出“为党育人、为国育才”的教育本质要求。

    东北大学的发展历程可以划分为几个重要阶段:在民族危亡的救国时期,学校师生以教育和科技救国为己任,展现了强烈的家国情怀;在建国初期,学校迅速投身社会主义建设,为国家工业化和国防事业输送了大批急需人才;在改革开放后,学校紧随国家发展战略,积极推动科技创新与人才培养;进入新时代,东北大学更加注重服务国家战略需求,培养创新型人才,彰显了其在国家发展中的独特地位。
\clearpage % 清除浮动,另起一页
\clearpage
\section{习近平总书记教育思想与大学使命}
    \label{sec:thir}
   教育兴则国家兴,教育强则国家强。习近平总书记多次强调,教育是民族振兴、社会进步的重要基石,是中华民族伟大复兴的基础性、战略性支撑\upcite{ref2}。大学作为高等教育的核心机构,在人才培养、科技创新、文化传承与社会服务等方面承担着不可替代的作用。

    \subsection{教育的战略地位}
    新时代背景下,我国正处于实现“两个一百年”奋斗目标的历史交汇期。总书记明确提出,要加快建设教育强国,办好人民满意的教育,为全面建设社会主义现代化国家提供有力支撑\upcite{ref3}。这不仅要求大学提升教育质量,更要面向国家重大战略需求培养拔尖创新人才。

    \subsection{中国特色世界一流大学建设}
    总书记指出,中国特色世界一流大学建设不是简单模仿国外模式,而是要坚持立德树人、服务国家的根本任务,走符合中国国情的高等教育发展道路\upcite{ref4}。这为包括东北大学在内的高校指明了前进方向,即要在传承红色基因、弘扬爱国主义精神的同时,注重创新能力与实践能力培养,实现人才质量与国家需求的紧密对接。

    \subsection{立德树人的根本任务}
    总书记强调,“立德树人”是教育的根本任务\upcite{ref5}。在这一理念指导下,大学不仅要注重知识传授,更要注重价值观引领和人格塑造,使学生成为德智体美劳全面发展的社会主义建设者和接班人。这对东北大学等具有爱国主义传统的高校提出了更高要求,即要以厚重校史和红色精神为依托,培养具有责任感、使命感和创新精神的新时代青年。
\clearpage % 清除浮动,另起一页
\section{东北大学案例与新时代启示}
    \label{sec:four}
    习近平总书记的回信不仅是对东北大学百年发展的总结,更是对其未来发展的殷切希望。结合新时代教育改革与国家发展需求,可以得到以下几点启示
    \subsection{坚持以国家战略需求为导向}
    高校的发展不能脱离国家战略大局。东北大学在百年历程中始终紧扣国家发展需要,这一经验对于新时代具有普遍意义。当前,服务东北全面振兴、服务国家重大战略是东北大学的历史使命。高校必须加强学科建设与国家战略的结合,在关键核心技术、能源资源开发、智能制造等领域做出更多贡献。

    \subsection{强化爱国主义教育与创新精神培养}
 东北大学素有爱国主义传统,这为新时代立德树人提供了坚实基础。高校要把爱国主义教育贯穿人才培养全过程,使学生树立家国情怀。同时,要注重培养创新精神与实践能力,推动人才成为科技强国和现代化建设的中坚力量。

    \subsection{推动人才培养模式改革}
    新时代人才需求多样化,传统的单一知识传授已不足以满足社会发展需要。高校要深化产学研融合,推动跨学科人才培养模式改革,注重学生综合素质和自主创新能力的提升,使人才更好适应国家现代化建设的需要。

    \subsection{传承与发展相结合}
    东北大学在历史中形成了鲜明的办学特色和文化传统,这些精神财富是新时代发展的重要资源。高校要善于将传统优势与现代需求结合起来,在保持特色的基础上不断创新发展。
\clearpage % 清除浮动,另起一页
\section{新时代大学生的责任与担当}
    \label{sec:five}
    习近平总书记强调,“青年兴则国家兴,青年强则国家强”\upcite{ref6}。作为新时代的大学生,既是学习者和继承者,更是未来国家建设和民族复兴的参与者与推动者。东北大学的学子在总书记回信精神的鼓舞下,应当从以下几方面明确自身责任与担当
    \subsection{坚定理想信念,厚植家国情怀}
    新时代大学生要把个人理想与国家发展紧密结合,坚定不移走中国特色社会主义道路,将自身成长融入中华民族伟大复兴的实践。

    \subsection{勇担科技创新使命}
    当前,国家正大力推进自主创新与科技自立自强。大学生要立志成为科技创新的主力军,积极投身科研攻关和技术突破,为国家解决“卡脖子”问题贡献智慧和力量。

    \subsection{提升综合素质与能力}
    大学教育不仅是知识的积累,更是能力的培养。新时代大学生要全面提升思想道德素质、科学文化素质和身心健康水平,做到德智体美劳全面发展,以适应未来社会的多元需求。

    \subsection{服务社会,勇于实践}
    青年学生要积极走向社会,参与志愿服务、社会调研和创新创业实践,在服务社会和解决实际问题中锻炼自己、提升自己,把个人成长与社会进步紧密结合。
\clearpage % 清除浮动,另起一页
\section*{总结}
    \addcontentsline{toc}{section}{总结}
    \label{sec:all}
    东北大学百年校庆之际,习近平总书记的重要回信既是对学校历史贡献的充分肯定,也是对其未来发展的深切期望\upcite{ref1}。通过对回信精神的学习和理解,可以明确大学教育与国家发展的内在联系。东北大学百年历程表明,高等教育的发展必须紧扣国家命运和民族复兴,才能实现自身价值。

    在新时代,大学要继续坚持“为党育人、为国育才”的根本任务,牢牢把握立德树人这一教育核心,服务国家战略需求,培养更多德才兼备、勇于创新的高素质人才。大学生则应当肩负起新时代的使命,将个人理想融入国家发展,把青春奋斗融入民族复兴伟业。唯有如此,才能真正实现教育强国、人才强国和科技强国的战略目标,为全面建设社会主义现代化国家提供坚强支撑。
% 参考文献
\clearpage % 清除浮动,另起一页
\begin{thebibliography}{99}
    \addcontentsline{toc}{section}{参考文献}
    \setlength{\itemsep}{0.5em} % 参考文献条目间距
    \songti\zihao{-4} % 确保参考文献部分也是宋体小四
    
\bibitem{ref1}
    新华社. 习近平给东北大学全体师生回信[N]. 人民日报. 2023-09-16.

    \bibitem{ref2}
    习近平. 在全国教育大会上的讲话[N]. 人民日报. 2018-09-11.
\bibitem{ref3}
    新华社. 习近平:加快建设教育强国,为中华民族伟大复兴提供有力支撑[N]. 人民日报,2023-05-29.

    \bibitem{ref4}
    习近平. 在北京大学师生座谈会上的讲话[N]. 人民日报,2018-05-03.

    \bibitem{ref5}
    中共中央文献研究室. 习近平关于教育的重要论述[M].
北京:中央文献出版社,2020.

    \bibitem{ref6}
    习近平. 在中国共产党第二十次全国代表大会上的报告[M]. 北京:人民出版社,2022.
\end{thebibliography}

\end{document}